\documentclass{beamer}
\usepackage{caption}
\captionsetup[table]{labelformat=empty}% removes label "Table:"
\usepackage{animate}
\usepackage{booktabs}
\usepackage{hyperref}

% try some other themes:
%\usetheme{Berkeley}

%\usetheme{Madrid}
\usetheme{Hannover}
% See a list of themes as https://hartwork.org/beamer-theme-matrix/

%\usecolortheme{beaver}
%\usecolortheme{crane}
%\usecolortheme{orchid}

\usefonttheme{serif}


% \usepackage{default}
% \usepackage{graphics}
% \usepackage{amsmath,amssymb,amsfonts,color,hyperref,bibentry,etex,xr}
% \usepackage[mathscr]{euscript}
% \usepackage[english]{babel}
% \usepackage[latin1]{inputenc}
\usepackage{graphics,bibentry,bm,lipsum,multimedia,media9}


%\usepackage{beamerthemesplit} %Activate for custom appearance


\usepackage[backend=biber,style=numeric,sorting=none]{biblatex}
\addbibresource{refs.bib}


%\hypersetup{colorlinks} % you can change the options to change link colors
 
\hypersetup{
    colorlinks=true,
    linkcolor=blue,
    citecolor=blue,
    filecolor=magenta,      
    urlcolor=cyan,
    %pdftitle={Overleaf Example},
    pdfpagemode=FullScreen,
    } 
   
\input{newcommands_Beamer.tex}    
% you can put a short tiel in brackets []
\title{Bias-Correcting Daily Satellite-Retrieved AOD For Air Quality Research}
\author{Wyatt G.\relax ~Madden\textsuperscript{$\dagger$}, Yang\relax ~Liu\textsuperscript{$\ddagger$}, Howard H.\relax ~Chang\textsuperscript{$\dagger$}}

\institute{\textsuperscript{$\dagger$}Department of Biostatistics \& Bioinformatics, Emory University,\\
\textsuperscript{$\ddagger$}Department of Environmental Health, Emory University\\}


\date{6 October 2023} % Date, can be changed to a custom date

% \let\newblock\relax
% \def\newblock{\hskip .11em plus .33em minus .07em}

\begin{document}

\frame{\titlepage}

\section{Introduction}

% note the use of the numbers after \item that is one way to progress through bullet points
\begin{frame}{Outline}

\begin{enumerate}
\item Background
\item Model 
\item Case Study 
\item \texttt{grmbayes} R Package
\end{enumerate}
\end{frame}


\section{Background}

\begin{frame}{Motivation - $PM_{2.5}$}

\begin{figure}[h]
    \includegraphics[width=\textwidth]{figures/atl_skyline_20180728_getty_images.png}
    \vspace{-0.5cm}
    \setbeamerfont{caption}{size=\scriptsize}
    \caption{Atlanta Skyline (Getty Images).}
\end{figure}



\end{frame}

\begin{frame}{Measure $PM_{2.5}$}
    \begin{figure}
        \begin{minipage}{0.48\textwidth}
            \includegraphics[width=1\linewidth, angle=270]{figures/Gillette_SLAMS_wyoming_dep_of_env_qual.jpg}
%    \vspace{-0.5cm}
            \setbeamerfont{caption}{size=\scriptsize}
            \caption{$PM_{2.5}$ Monitoring Station (WDEQ).}
        \end{minipage}
        \hfill
        \begin{minipage}{0.48\textwidth}
            \includegraphics[width=1\linewidth]{figures/obs_map_20181008.png}
%    \vspace{-0.5cm}
            \setbeamerfont{caption}{size=\scriptsize}
            \caption{AQS $PM_{2.5}$ Oct 8th, 2018}
        \end{minipage}
    \end{figure}
\end{frame}

\begin{frame}{Fill in the gaps - CTM}
    \begin{figure}
        \begin{minipage}{0.48\textwidth}
            \includegraphics[width=0.8\linewidth]{figures/CMAQ_25th_logo.png}
%    \vspace{-0.5cm}
            \setbeamerfont{caption}{size=\scriptsize}
            \caption{epa.gov}
        \end{minipage}
        \hfill
        \begin{minipage}{0.48\textwidth}
            \includegraphics[width=1\linewidth]{figures/ctm_map_20181008.png}
%    \vspace{-0.5cm}
            \setbeamerfont{caption}{size=\scriptsize}
            \caption{CMAQ Simulation Oct 8th, 2018}
        \end{minipage}
    \end{figure}
\end{frame}

\begin{frame}{Fill in the gaps - Aerosol Optical Depth}
    \animategraphics[loop,controls,width=0.5\linewidth]{12}{figures/maia_gif/maia-}{0}{50}
\end{frame}

\section{Model}

\begin{frame}{Big Picture}
    \begin{enumerate}
        \item Match Monitor $PM_{2.5}$ to nearest CTM
        \item Regress Monitor $PM_{2.5}$ on matched CTM
        \item Predict $PM_{2.5}$ for all CTM with regression model
    \end{enumerate}
    \begin{figure}
        \begin{minipage}{0.48\textwidth}
            \includegraphics[width=1\linewidth]{figures/obs_map_20181008.png}
%    \vspace{-0.5cm}
            \setbeamerfont{caption}{size=\scriptsize}
            \caption{AQS $PM_{2.5}$ Oct 8th, 2018}
        \end{minipage}
        \hfill
        \begin{minipage}{0.48\textwidth}
            \includegraphics[width=1\linewidth]{figures/ctm_map_20181008.png}
%    \vspace{-0.5cm}
            \setbeamerfont{caption}{size=\scriptsize}
            \caption{CMAQ Simulation Oct 8th, 2018}
        \end{minipage}
    \end{figure}
\end{frame}

\begin{frame}{Geostatistical Regression Model}
    \begin{align*} 
        PM_{2.5}(s,t) = \alpha_0(s, t) + \alpha_1(s, t) CTM(s, t) + \gamma Z + \epsilon(s, t)
    \end{align*}
    \begin{align*}
        \alpha_0(s, t) = \beta_0(s) + \beta_0(t)  \\
        \alpha_1(s, t) = \beta_1(s) + \beta_1(t)
    \end{align*} 


    where $\beta_i(s) \sim GP(0, \tau_i^2 K_i)$, $\beta_i(t)$ are modeled as first-order random walks, and $\gamma$ are fixed effects for spatial or spatio-temporal varying covariates $Z_i$. All priors are weakly-informative. \cite{chang2014}
\end{frame}

\begin{frame}{Gaussian Process}

        $\beta(s) \sim GP(0, \tau^2 K)$, 
    $K(s, s') = \exp\left(-\frac{||s - s'||}{\theta}\right)$,\\

    so for given set of $\beta$, $\beta(s) \sim MVN(0, \tau^2 K)$


\begin{figure}[h]
    \includegraphics[width=0.6\textwidth]{figures/gp_sim.png}
    \vspace{-1cm}
    \setbeamerfont{caption}{size=\scriptsize}
\end{figure}

\end{frame}

\begin{frame}{GP to NNGP}
    Inverting $K$ is $O(n^3)$, so GP too slow for many monitors. \\
    \vspace{0.5cm}
    In all cases:
    \begin{align*}
        p(\beta(\pmb{s})) &= p(\beta_1, \dots, \beta_n) \\
                          &= p(\beta_1 | \beta_2, \dots, \beta_n) p(\beta_2 | \beta_3, \dots, \beta_n) \dots p(\beta_{n-1} | \beta_n) p(\beta_n) \\
    \end{align*}
    and under NNGP\cite{datta2014} assumption:
    \begin{align*}
                          &= p(\beta_1 | \beta_2, \dots, \beta_{m + 1}) p(\beta_2 | \beta_3, \dots, \beta_{m + 1}) \dots p(\beta_{n-1} | \beta_n) p(\beta_n) \\
    \end{align*}
    resulting in $O(nm^3)$ inversion, where $m$ is number of neighbors. 

\end{frame}
\begin{frame}{NNGP Neighbors}
    \animategraphics[loop,controls,width=0.7\linewidth]{2}{figures/nngp_gif/nngp-}{0}{50}
\end{frame}


\section{Case Study}

\begin{frame}{Data - 2018 U.S.}
    \begin{itemize}
        \item All days in 2018
        \item 973 AQS monitors 
        \item 55,508 12km gridded CMAQ simulation locations
        \item $n = 973 \text{ x } 365 = 355,145$, $n_{pred} = 55,508 \text{ x } 365 = 20,260,420$
        \item High variance due to Californian fires
    \end{itemize}

    \begin{table}
        \centering
        \resizebox{0.5\textwidth}{!}{
            \input{tables/obs_monthly_stats.tex}
        }
        \caption{Monthly summary statistics for AQS $PM_{2.5}$ in 2018.}
    \end{table}

\end{frame}


\begin{frame}{Performance - 2018 Just California}

    \begin{table}
        \centering
        \input{tables/cv_rmse_just_ca.tex}
        \caption{ California in-sample RMSE for all combinations of spatial process and $\theta$ discretization schemes.}
    \end{table}


\end{frame}

\begin{frame}{Performance - 2018 All U.S.}

    \begin{table}
        \centering
        \resizebox{0.8\textwidth}{!}{
            \input{tables/cv_rmse_full_grid.tex}
        }
        \caption{Full U.S. RMSE (95\% Prediction Interval Coverage Probability) for all combinations of CV type and Mat\'{e}rn $\nu$ parameter values.}
    \end{table}


\end{frame}

\begin{frame}{Prediction - Oct 8th, 2018}

\begin{figure}[h]
    \includegraphics[width=0.9\textwidth]{figures/pred_map_20181008.png}
    \vspace{-1cm}
    \setbeamerfont{caption}{size=\scriptsize}
\end{figure}

\end{frame}

\begin{frame}{Prediction - Posterior Mean By Season}
\begin{figure}[h]
    \includegraphics[width=0.9\textwidth]{figures/pred_map_season.png}
    \vspace{-1cm}
    \setbeamerfont{caption}{size=\scriptsize}
\end{figure}

\end{frame}


\section{R Package}

\begin{frame}{\texttt{grmbayes} R Package}
    {\tiny
    \begin{itemize}
        \small
        \item \textbf{Spatial Process}: Select either GP or NNGP (with $m$ number of neighbors)
        \item \textbf{Random Effects}: Select either additive or multiplicative random effects for spatial and/or temporal components.
        \item $\pmb{\theta}$ \textbf{Discretization}: Discretize the spatial process range parameter $\theta_i$ for spatial intercept and/or spatial slope. Choose levels, and either Gibbs or Metropolis-Hastings updating schemes. 
        \item \textbf{Cross Validation}: Choose number of folds, and cross validation type (out of `ordinary', `spatial', `spatial clustered' or `spatial buffered' with a corresponding buffer size). 
        \item \textbf{Covariance Kernal}: Select Mat\'{e}rn($\theta$, $\nu$) covariance function with $\nu \in \{\frac{1}{2}, \frac{3}{2}, \frac{5}{2} \}$, or input user-defined covariance kernal.
        \item \textbf{Covariates}: Include additional regression covariates. 
    \end{itemize}
    }
\end{frame}

\begin{frame}{\texttt{grmbayes} tutorial}
    \url{https://github.com/WyattGMadden/grmbayes}
\end{frame}

\begin{frame}{Next Steps}   
    \begin{enumerate}
        \item Include spacetime random effects for case study 
        \item Submit package to CRAN
        \item Spatially varying covariates
        \item Anisotropy/non-stationarity 
        \item Spectral embeddings
        \item Variable selection
    \end{enumerate}

\end{frame}

\section{References}


\begin{frame}[allowframebreaks]
\frametitle{References}
\vspace{-1em}
\printbibliography[heading=none]
\end{frame}
   
\end{document}
