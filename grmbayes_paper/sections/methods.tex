
\section*{Methods}

\subsection*{Model}

The geostatistical regression model is formulated as a spatial-temporal regression of $Y = pm_{2.5}$ against $X$, which is either AOD or CTM depending on user input. 

The statistical model is as follows:

\begin{align*} 
    pm_{2.5}(s,t) = \alpha(s, t) + \beta(s, t) X(s, t) + \mathbf{l}(s)' \bs{\gamma} +  \mathbf{m}(s, t)' \boldsymbol{\delta} + \epsilon(s, t)
\end{align*}
         
where $\alpha_0(s, t)$ and $\beta(s, t)$ are the intercept and AOD/CTM slope of the regression model at location $s$ and time $t$, $\bs{\gamma}$ and $\bs{\delta}$ are fixed effects for spatial and spatio-temporal covariates $\mathbf{l}(s)$ and $\mathbf{m}(s, t)$ respectively, and $\bs{\epsilon}(s, t) \sim N(0, \sigma^2)$. 

The slope and intercept parameters are composed of the following spatial and temporal random effects and fixed effects:

\begin{align*}
    \alpha(s, t) = \alpha_0 + \alpha(s) + \alpha(t) \\
    \beta(s, t) = \beta_0 + \beta(s) + \beta(t) \\
\end{align*} 

where spatial random effects $\bs{\alpha}(\bl{s}), \bs{\beta}(\bl{s}) \sim SP(\bl{0}, \tau^2 \bl{K}(\theta, d))$ for a user specified stochastic process $SP$ and kernal $\bl{K}$ that depends on range parameter $\theta$ and distance $d$, temporal random effects $\bs{\alpha}(\bl{t}), \bs{\beta}(\bl{t})$ assumed first-order random walk.
 Normal priors are applied to fixed effects $\bs{\gamma}$ and $\bs{\delta}$ with equivalence to a ridge regression shrinkage term. 

Users can specify inclusion of any combination of additive or multiplicative spatial or temporal random effects, as well as input $\bl{L}$ and $\bl{M}$ matrices for fixed effects.
For example, if the user-specifies inclusion of an additive spatial effect, a multiplicative temporal effect, and no $\bl{L}$ and $\bl{M}$ matrices, the intercept/slope equations would simplify as follows:

\begin{align*}
    \alpha(s, t) =& \alpha_0 + \alpha(s)\\
    \beta(s, t) =& \beta_0 + \beta(t) \\
\end{align*} 



\subsection*{Spatial Random Effects}

The spatial random effects are modeled as either a Gaussian Process (GP) or Nearest Neighbor Gaussian Process (NNGP) as specified by user input. 
The NNGP is a stochastic process that assumes the joint distribution of spatial effects can be decomposed into conditional distributions that depend only on a specified number of nearest neighbors. 
This allows for the covariance kernal to be inverted in $\mathcal{O}(n m^3)$ time, where $n$ is the number of spatial effects and $m$ is the number of neighbors, as opposed to $\mathcal{O}(n^3)$ for a regular GP.
The NNGP approximates a GP when the number of neighbors is sufficiently large (citation for specific $m$), and often results in similar estimates and predictive performance. 

We provide four covariance kernals for the spatial random effects which are available for both GP and NNGP specifications: exponential, and Mat\'{e}rn for $\nu \in {\frac{1}{2}, \frac{3}{2}, \frac{5}{2}}$, where:
\begin{align*}
    \bs{K}_{exp} =& \exp \left( - \frac{d}{\theta} \right) \\
    \bs{K}_{mat} =& \frac{1}{\Gamma(\nu) 2^{\nu - 1}} \left( \frac{\sqrt{2 \nu}}{\theta} d \right)^{\nu} K_{\nu} \left( \frac{\sqrt{2 \nu}}{\theta} d \right) \\
\end{align*} 

We also allow for the user to specify a custom covariance kernal, which must be positive definite and symmetric and specified with respect to the range parameter $\theta$ and distance $d$ such that $\bl{K}_{user} = f(\theta, d)$. 

Another method for improving computational efficiency is to discretize the range parameters $\theta_{\alpha}$ and $\theta_{\beta}$, allowing for pre-computation of kernal inverses that are recycled at each MCMC iteration.
Users can inpute discrete values for each range parameter, as well as specify the MCMC step for the range parameter as either Metropolis-Hastings (MH) or Gibbs. 
The MH step requires the computation of two likelihoods (with respect to both the current and the proposal $\theta$) at each iteration, as opposed to Gibbs which requires computation of a likelihood for every discrete $\theta$ value for each iteration. 
However MH may result in poorer mixing, thus requiring more iterations to converge.

Finally, while spatio-temporal effects are not currently implemented, we do provide the option to use different sets of spatial effects for different time periods. 
By using different spatial effects for say seasons, or months, some temporal variation in spatial effects can be accounted for.
Regardless of number of spatial effect sets, spatial stochastic process parameters ($\tau^2$, $\theta$) are shared across sets. 
For example, if spatial-time sets are specified for seasons, the additive spatial random effect is specified as follows:
\[
    [\bs{\alpha}_n(\bl{s}_{spr}), 
        \bs{\alpha}_n(\bl{s}_{sum}),
        \bs{\alpha}_n(\bl{s}_{fal}),
        \bs{\alpha}_n(\bl{s}_{win})]' \sim SP(\bl{1}_4 \otimes \bl{0}_n, \tau^2 \bl{I}_{4,4} \otimes \bl{K}(\theta, d)_{n,n})
\]
where there are $n$ spatial locations and $\otimes$ is the Kronecker product.

\subsection*{Temporal Random Effects}

The first-order random walk temporal random effects ($\bs{\alpha}(\bl{t}), \bs{\beta}(\bl{t})$) are specified such that,

\begin{align*}
    E[\alpha(t)] =& 
    \begin{cases} 
        \rho_{\alpha} \alpha(t+1) & \text{if } t = 1 \\ 
        \rho_{\alpha} \frac{\alpha(t-1) + \alpha(t+1)}{2} & \text{if } 1 < t < T \\
        \rho_{\alpha} \alpha(t-1) & \text{if } t = T \\
    \end{cases} \\
\end{align*}

\begin{align*}
    Var[\alpha(t)] =& 
    \begin{cases} 
        \omega^2_{\alpha} & \text{if } t = 1 \\
        \frac{\omega^2_{\alpha}}{2} & \text{if } 1 < t < T \\
        \omega^2_{\alpha} & \text{if } t = T
    \end{cases} \\
\end{align*}

with $\bs{\beta}(\bl{t})$ similarly specified. 
Here each $\rho$ is discretized as $2,000$ evenly spaced values between $0$ and $1$, and each $\omega$ determines the temporal smoothness level. 

\subsection*{Fixed Effects}


\subsection*{Cross-Validation}

We provide functions to perform k-fold cross-validation with the geostatistical regression model.
K-fold cross-validation prevents overfitting by separating the data set into k number of folds, iteratively fitting the model to k-1 folds and predicting the remaining fold. (INCLUDE citation)
The user can specify the number of folds, the type of cross-validation to perform, and any related parameters for the type of cross-validation specified.
We provide the following types of cross-validation:

\begin{itemize}
  \item \textbf{Ordinary}: Folds are randomly assigned across all observations
  \item \textbf{Spatial}: Folds are randomly assigned across all spatial locations. 
  \item \textbf{Spatial Clustered}: K spatial clusters are estimated using k-means clustering on spatial locations. These clusters determine the folds. 
  \item \textbf{Spatial buffered}: Folds are randomly assigned across all spatial locations. For each fold, observations are removed from the training set if they are within a user-specified distance from the nearest test set point. 
\end{itemize}

When assigning folds we ensure that each fold contains at least one observation from each spatial location.
If there are fewer observations than folds for a given spatial location we leave data from this location unassigned. 
Data from the first and last time point are also unassigned (INCLUDE why?).
Predictions are output as missing for unassigned data.

INCLUDE figure detailing cv types




























