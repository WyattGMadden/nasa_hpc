
\section*{Introduction}

Air pollution exposure is strongly associated with negative health outcomes. 
Particulate matter of size less than 2.5 $\mu$m in diameter ($PM_{2.5}$) has been shown to be an especially harmful component of air pollution. 
For this reason, precise estimation and prediction of $PM_{2.5}$ is an important component of air pollution health research.  

There exists a network of monitors that provide precise, direct measurements of $PM_{2.5}$. 
However these monitors are relatively expensive to install and maintain, and therefore are both spatially sparse and preferentially located in prioritized areas. 

There exist two additional data sources that are highly correlated with $PM_{2.5}$, enabling the development of statistical methods to predict $PM_{2.5}$ beyond locations at which monitors are located. 
The first is Chemical Transport Models (CTMs) which provide numerical model simulations of air quality based on physical processes and meteorological data. 
The second is satellite measurements of Aerosol Optical Depth (AOD), which provide a noisy but fine-grained measurement of how much sunlight reaches the ground. 



An important area of statistical research is the development of methods that can use either one or both of these measurements to predict $PM_{2.5}$ for locations at which monitors are not located. 
One such method combines regression models from each of these data sources using Bayesian ensemble averaging, allowing both predictions based on available data for a given grid cell in space and time, and full Bayesian uncertainty quantification \cite{murray2019}. 
Here we introduce an R package, \texttt{ensembleDownscaleR}, containing a suite of functions that allow users to predict $PM_{2.5}$ for large regions of space and time, using this method.

In this work we first reintroduce the statistical method from \cite{murray2019} with the inclusion of additional specifications that were not detailed in the original manuscript. 
We then detail the functionality of the R package and provide an example analysis using 2018 data from the Los Angeles metropolitan area. 



